Прикладные задачи часто формулируются так, что для них нет одного оптимального решения, и необходимо руководствоваться несколькими критериями. Для решения подобных задач есть методы многокритериального принятия решений и методы многокритериальной оптимизации \cite{koksalan2011multiple}. Задачи из этой области и методы их решения находят широкое применение в различны областях. Примеры находятся в задачах выбора параметров в сети электропитания и телекоммуникаций \cite{altiparmak2006genetic, elmusrati2008applications, mastrocinque2013multi, bjornson2014multiobjective}, задачах машинного обучения \cite{suttorp2006multi, zuluaga2013active, sener2018multi}, химии \cite{rangaiah2013multi}, биологии \cite{boada2016multi}, и задачах  из инженерных областей \cite{marler2004survey}.  В многокритериальной оптимизации распространены два подхода -- аппроксимация парето фронта и скаляризация задачи, то есть сведение задачи к задаче с одним критерием оптимальности.

Для аппроксимации парето фронта существует множество методов: методы на основе генетических и эволюционных подходов  \cite{ngatchou2005pareto, konak2006multi}. Такие алгоримы не эффективны по количеству семплов и являются вычислительно дорогими.
Интерактивные подходы \cite{miettinen2008introduction}. Как и сказано в названии класса методов, для их использования требуется участие внешнего эксперта в виде человека.
Появляются работы основанные на байесовском подходе \cite{suzuki2020multi,daulton2022multi} и на восстановлении направлений убывания функций \cite{gebken2021efficient}. Авторы байесовского подхода отмечают значительный прирост эффективности алгоритма по семплам по сравнению с предшествующими алгоритмами. 


Методы сведения задачи к задаче одноцелевой оптимизации также широко распространены. Популярными являются методы взвешивания: линейное взвешивание, взвешенная $t$-ая степень, взвешенная квадратичная задача, $\epsilon$ ограничивающий подход. Семейство методов с целевой точкой: метод на основе функции расстояний, функции достижимости и другие. Семейство методов, основанных на направлениях: подходы Пасколетти, Серафини, Ф. Гембички(F Gembicki) и остальные. Названные методы в подробностях разобраны в книге \cite{greco2006multiple}. Множество книг посвящено теме многокритериальной оптимизации \cite{miettinen1999nonlinear,greco2006multiple,koksalan2011multiple}.

Методы восстановления парето фронта используются для подробного изучения оптимальных параметров задачи. Однако для современных задач подобное удовольствие слишком дорогое. Поэтому представляют интерес методы, позволяющие быстро находить набор параметров с определенным свойством. По сути методы скаляризации ставят задачу такого поиска. Например, методы взвешивания задают приоритет на оптимизируемых функциях. Однако, по мере изучения методов скаляризации становится ясно, что методы требуют введения необучаемых параметров. От этих параметров зависит сложность поиска решения. В методах взвешивания такими параметрами являются веса, с которыми берутся функции. Ещё одним недостатком является неинтерпретируемость полученного решения. Эту неопределенность помогает решить идея конкурентного (competitive) решения. Она дает хорошо интерпретируемое на практике решение без введения большого количества параметров. Однако и у нее есть свои недостатки. В этой работе предлагается расширить определение конкурентного решения и на основе этого расширения ставится задача оптимизации- метод скаляризации. Для липшицевых функций предлагается вычислительно эффективный метод поиска приближенного конкурентного решения. Метод предполагается использовать в случае сильного ограничения в вычислительных ресурсах и когда нет возможности повторно вычислять функции. Итеративная оптимизация в обоих случаях недоступна. Это актуально, поскольку современные задачи имеют большие размерности и градиентные и эволюционные методы для них неэффективны или даже неприменимы.

Дальнейший текст составлен следующим образом: в \ref{sec:task_statement} Определяется задача оптимизации, вводится определение конкурентного решения и приводится обощение этого понятия. На основании этого определения ставится задача оптимизации и отмечается связь этой постановки с работами ранее. В разделе \ref{sec:method} вводится предлагаемый метод решения задачи для Липшицевых функций и в разделе \ref{sec:theorems} доказываются некоторые утверждения для предложенных методов. В разделе \ref{sec:experiments} приводятся численные эксперименты для демонстрации работы методов.